\documentclass[a4paper,10pt]{article}

% --- Configuración básica ---
\usepackage[utf8]{inputenc}
\usepackage[T1]{fontenc}
\usepackage[spanish,es-lcroman]{babel}
\usepackage{geometry}
\geometry{margin=1in}
\usepackage{xcolor}
\usepackage{hyperref}
\usepackage{enumitem}
\usepackage{parskip}
\usepackage{titlesec}

% --- Estilo visual ---
\definecolor{primary}{HTML}{6C63FF}
\definecolor{textgray}{HTML}{2B2B2B}
\definecolor{muted}{HTML}{666666}
\hypersetup{
  colorlinks=true,
  linkcolor=primary,
  urlcolor=primary
}

% --- Títulos y formato ---
\titleformat{\section}{\large\bfseries\color{primary}}{}{0em}{}
\titleformat{\subsection}{\normalsize\bfseries\color{textgray}}{}{0em}{}
\setlist[itemize]{leftmargin=1.2em, itemsep=0.4em} % separación entre bullets
\titlespacing*{\section}{0pt}{1.5em}{0.8em} % espacio antes/después de sección
\titlespacing*{\subsection}{0pt}{1em}{0.5em} % espacio entre subsecciones

\setlength{\parskip}{6pt} % espacio entre párrafos
\pagestyle{empty}

\begin{document}

% ================== ENCABEZADO ==================
\begin{center}
  {\LARGE \textbf{Ingrid Jacqueline Aravena Barra}}\\[4pt]
  {\color{muted}Ingeniera Comercial \,|\, Relatora y Facilitadora \,|\, IA, ML y Marketing Digital}\\[2pt]
  \href{mailto:iaravena24@gmail.com}{iaravena24@gmail.com} \,|\, 
  \href{https://github.com/iaravena24}{github.com/iaravena24} \,|\, Concepción, Región del Biobío
\end{center}

\vspace{1.2em}

% ================== PERFIL ==================
\section*{Perfil Profesional}
Ingeniera comercial con experiencia en docencia, capacitación y gestión de proyectos en ámbitos administrativo, social, universitario, ventas y formación e\,-learning. Fortalezas: autonomía, visión e investigación, planificación y orden, atención al detalle y solución de problemas.

% ================== FORMACIÓN ==================
\section*{Formación}
\textbf{Ingeniera Comercial, Licenciada en Ciencias Administrativas} — Universidad del Bío Bío, 2007\\
\textbf{Diplomado en Gestión de RR.HH} — Universidad de Concepción, 2014

\vspace{0.8em}
\textbf{Cursos y certificaciones recientes}\\
Google AI Essentials (Coursera, 2024); Fundamentos de Marketing Digital y Comercio Electrónico (Coursera, 2024); 
Atraer clientes e interactuar mediante marketing digital (Coursera, 2024); 
IA Generativa para el día a día: conceptos básicos y responsabilidad ética (U. de Concepción, 2025);
Gestión del tiempo y productividad (SENCE, 2025); 
Educación para el consumo sostenible en la escuela (SERNAC, CPIP, 2025, 200 h);
Especialización en Machine Learning (en curso, SENCE–Talento Digital, 2025); 
Introducción a la ciencia de datos (Santander Open Academy / IE University, 2025, 6 h).

% ================== EXPERIENCIA RECIENTE ==================
\section*{Experiencia Profesional (2020–2025)}

\vspace{0.4em}
\subsection*{Relatorías y Facilitación}
\textbf{OTEC CENFOR} — Relatora \hfill Sept.–Nov. 2025\\[-0.3em]
\begin{itemize}
  \item Administración de Redes Sociales (2 cursos, 140 h en total).
  \item Gestionando y Formalizando mi Emprendimiento (61 h, e\,-learning).
\end{itemize}

\vspace{0.8em}
\textbf{Fundación Emplea} — Relatora \hfill Nov. 2023–Ago. 2025\\[-0.3em]
\begin{itemize}
  \item Técnicas de Planificación y Administración de Negocios (5 cursos, 360 h).
  \item Educación Financiera (18 cursos, 540 h).
  \item Educación Financiera para Emprendedores (3 cursos, 90 h).
\end{itemize}

\vspace{0.8em}
\textbf{OTEC Talento Local} — Relatora \hfill Jul.–Ago. 2025\\[-0.3em]
\begin{itemize}
  \item Community Manager y Marketing Digital (2 cursos, 27 h c/u).
\end{itemize}

\vspace{0.8em}
\textbf{Fundación FORPE} — Relatora \hfill Jul. 2025\\[-0.3em]
\begin{itemize}
  \item Gestionando y Formalizando mi Emprendimiento (presencial).
\end{itemize}

\vspace{0.8em}
\textbf{OTEC CAPACITY (empresa CALLFIRE)} — Relatora \hfill Jul. 2025\\[-0.3em]
\begin{itemize}
  \item Gestión comercial para equipos no especializados (16 h).
\end{itemize}

\vspace{0.8em}
\textbf{OTEC DELPHINUS} — Apoyo Socio Laboral \hfill Jun.–Jul. 2025\\[-0.3em]
\begin{itemize}
  \item Curso “Gestionando mi Emprendimiento” (Comunas de Lota y Concepción).
\end{itemize}

\vspace{0.8em}
\textbf{OTEC PROBIOBIO} — Relatora \hfill Jun. 2025\\[-0.3em]
\begin{itemize}
  \item Operador Venta Asistida Perecibles (60 h).
\end{itemize}

% ================== HABILIDADES ==================
\vspace{1em}
\section*{Habilidades}
\textbf{Formación y Docencia:} relatoría e\,-learning, diseño de cursos, acompañamiento socio-laboral, formación de adultos.\\[3pt]
\textbf{Negocios y Marketing:} marketing digital, redes sociales, educación financiera, gestión comercial, formulación de proyectos.\\[3pt]
\textbf{Tecnología y Plataformas:} Teams, Zoom, Google Meet, BigBlueButton.\\[3pt]
\textbf{Idiomas:} Español (nativo); Inglés (TOEIC/Comunicacional, formación UBB/Tronwell).

% ================== CONTACTO ==================
\vspace{1em}
\section*{Contacto}
Para contacto profesional: \href{mailto:iaravena24@gmail.com}{iaravena24@gmail.com}

\end{document}
